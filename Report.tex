\documentclass[12pt]{report}

% Import packages
\usepackage{graphicx}  % For including images
\usepackage{amsmath}   % For advanced math formatting
\usepackage{hyperref}  % For creating hyperlinks
\usepackage{geometry}  % To set page margins
\usepackage{helvet}
\renewcommand{\familydefault}{\sfdefault}  % Makes sans-serif the default font
\geometry{a4paper, margin=1in}

\begin{document}

\title{\textbf{k Nearest Neighbors - kNN}}
\author{Savvas Tzanetis}
\maketitle  % Creates the title

\tableofcontents  % Generates the table of contents

\chapter{Introduction}
This project is part of an assignment in Parallel and distributed systems class of the Aristotle University of Thessaloniki. The assignment requires implementing a \textbf{k-NN} algorithm to find the\textbf{ k nearest neighbors} of a set of points from a Query data set, relative to a Corpus data set in a high-dimensional space. This algorithm, implemented in the \textbf{C++} language, given a set of M-points from a Corpus data set and N-points from a Query data set, both in the same D-dimensional space, identifies the k-nearest neighbors of each query point. Using optimized matrix operations as well as the \textbf{OpenBLAS} library, this algorithm handles high-dimensional distance calculations efficiently.

\section{Problem Analysis}
The objective of the assignment is to implement a subroutine in \textbf{C++} that computes the \textbf{k nearest neighbors} of each query point in \textbf{Q(Query set)} relative to a set of points in \textbf{C(Corpus set)} based on their distances, using a multitude of tools for implementing a parallel implementation as well as a version using serial computing.

\chapter{Serial Version V0}
In this chapter, we explain the methods used in the project.

\chapter{Parallel Version}
In this chapter, we present the results.

\section{Using OpenMP}
Description of the first experiment and results.

\section{Using OpenCILK}
Description of the first experiment and results.

\section{Using PTHREADS}
Description of the first experiment and results.

\chapter{Conclusion}
Summarize the findings and conclude the report.

\end{document}