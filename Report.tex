\documentclass[12pt]{report}

% Import packages
\usepackage{graphicx}  % For including images
\usepackage{amsmath}   % For advanced math formatting
\usepackage{bm}
\usepackage{hyperref}  % For creating hyperlinks
\usepackage{geometry}  % To set page margins
\usepackage{helvet}
\renewcommand{\familydefault}{\sfdefault}  % Makes sans-serif the default font
\geometry{a4paper, margin=1in}

\begin{document}

\title{\textbf{k Nearest Neighbors - kNN}}
\author{Savvas Tzanetis}
\maketitle  % Creates the title

\tableofcontents  % Generates the table of contents

\chapter{Introduction}
The k-Nearest Neighbors \textbf{(kNN)} algorithm is a fundamental technique for finding the closest neighbors of a query set \textbf{Q} with respect to a data corpus \textbf{C}. However, its computational complexity becomes prohibitive for large datasets, especially when the corpus and query set have high dimensions. This report presents a parallel implementation of the kNN search algorithm by leveraging parallelism using libraries like \textbf{OpenMP, OpenCILK} and \textbf{PTHREADS} as part of an assignment in the Parallel and distributed systems class of the department of Electrical and computer engineering of the Aristotle unversity of Thessaloniki. This algorithm, implemented in the \textbf{C++} language uses the \textbf{OpenBLAS} library for optimized matrix computations for high dimensional distance calculations and the correctness of the implementation is validated against MATLAB's \textbf{knnsearch}.

\section{Problem Analysis}
The kNN problem aims to compute the indices and distances of the k nearest neighbors for each query point in Q with respect to C. The exact computation involves:
\[
\bm{D} = \sqrt{\bm{C}^2 - 2\bm{C}\bm{Q}^T + \bm{Q}^2}
\]
where the operations are element-wise.
Due to memory constraints, the distance matrix D is computed in blocks by dividing Q into subsets, computing neighbors recursively for these subsets and finally, merging the partial solutions efficiently.

\chapter{Implementation Details}

\chapter{Parallelization Techniques}
The distance matrix D is computed in blocks, allowing memory-efficient operations and parallel computation. Each block is processed concurrently using different parallel paradigms.

\section{Using OpenMP}
\begin{itemize}
    \item A shared-memory approach to parallelize block computations.
    \item Loops are parallelized using \textbf{\#pragma omp parallel for}, distributing blocks across threads.
\end{itemize}

\section{Using OpenCILK}
\begin{itemize}
    \item Provides parallelism using lightweight tasks and cilkfor.
    \item Recursive computations (e.g., merging solutions) are naturally expressed using OpenCilk's divide-and-conquer model.
\end{itemize}

\section{Using PTHREADS}
\begin{itemize}
    \item Fine-grained control over thread creation and workload distribution.
    \item Each thread processes a distinct block of D, requiring explicit synchronization for merging results.
\end{itemize}

\chapter{Results and Correctness Verification}

\end{document}